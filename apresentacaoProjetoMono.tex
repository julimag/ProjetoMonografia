%\documentclass{beamer} % apresentacao

% That doesn't affect the result. There is only one less warning.
%\documentclass[hyperref={pdfpagelabels=false}]{beamer} % apresentacao sem warning de ''... \thepage is undefined''
\documentclass[hyperref={pdfpagelabels=false},smaller]{beamer}
%===============================================================================
% PACKAGES
%===============================================================================

% preambulo para escrever em portugues
\usepackage[brazil]{babel}
\usepackage[latin1]{inputenc}
\usepackage[T1]{fontenc}
% usar cor em tabela
\usepackage{colortbl}

\setbeamertemplate{navigation symbols}{}

%===============================================================================
% THEMES
%===============================================================================

\usetheme[secheader]{Berkeley} %side nav list
% \usepackage{graphicx}

% \mode<presentation>

%
% MACROS
%
\newcommand{\univ}{Faculdade 7 de Setembro - FA7}
\newcommand{\titulo}{Testes de Acessibilidade Automatizados para Aplicativos Web em Dispositivos M�veis}
\newcommand{\tipo}{Projeto de Monografia}
\newcommand{\autor}{Juliana Feitosa Magalh�es}
\newcommand{\curso}{Especializa��o em Desenvolvimento �gil de Software}
\newcommand{\coordenador}{Prof. Ciro Carneiro Coelho}
\newcommand{\disciplina}{Metodologia Cient�fica}
\newcommand{\professora}{Profa. Aline Mota Albuquerque}
\newcommand{\local}{Fortaleza}
\newcommand{\data}{Dezembro de 2011}

\begin{document}

\title[\tipo]{\titulo}
\author[]{\autor}
\institute[]{\univ \\ \curso \\ \coordenador \\ \disciplina \\ \professora}
\date[\data]{\data} % para fixar a data de acordo com o macro \data declarado

\frame{\titlepage}

%===============================================================================
\section{Introdu��o}

\begin{frame}

   \frametitle{Introdu��o}

      \begin{block}{O que faz parte do tema?}
         \begin{itemize}
            \item Dispositivos M�veis
            \item Aplicativos Web
            \item Testes de Acessibilidade
            \item Testes Automatizados
         \end{itemize}
      \end{block}

\end{frame}

%===============================================================================
\section{Problema}

\begin{frame}

  \frametitle{Problema}

  \begin{alertblock}{O que e como testar?}
    \begin{itemize}
      \item Quais as regras de acessibilidade para web?
      \item Como s�o feitos os testes de acessibilidade? E com quais ferramentas?
      \item Como automatizar testes de acessibilidade?
      \item Quais as restri��es de testes para dispositivos m�veis?
	\begin{itemize}
	  \item Tamanho da tela (tela reduzida)
	  \item Novas tecnologia (tela sens�vel ao toque)
	  \item Poder de processamento menor (avaliar performance)
	  \item Velocidade da internet dispon�vel (analisar desempenho)
	\end{itemize}
    \end{itemize}
  \end{alertblock}

\end{frame}

%===============================================================================
\section{Justificativa}

\begin{frame}

   \frametitle{Justificativa}

      \begin{block}{Porque ?!?}
         \begin{itemize}
            \item Porque testar?
            \item Porque acessibilidade?
            \item Porque automatizar?
            \item Porque dispositivos m�veis?
	    \item Porque web?
         \end{itemize}
      \end{block}

\end{frame}

%===============================================================================
\section{Objetivo}

\begin{frame}

   \frametitle{Objetivo Geral e Objetivo Espec�fico}

      \begin{block}{Geral}

         Estudar a problem�tica dos testes de acessibilidade no contexto m�vel.

      \end{block}

      \begin{block}{Espec�fico}

         Implementar testes automatizados de acessibilidade com foco em navegadores de dispositivos m�veis.

      \end{block}

\end{frame}

%===============================================================================
\section{Hip�tese}

\begin{frame}

   \frametitle{Hip�tese}

      As ferramentas de testes automatizados n�o s�o comumente utilizadas para verificar e validar �s restri��es de acessibilidade em dispositivos m�veis.

\end{frame}

%===============================================================================
\section{Metodologia}

\begin{frame}

   \frametitle{Metodologia}

  \begin{exampleblock}{}
    \begin{itemize}
      \item Elencar as principais regras de acessibilidade para web:
	\begin{itemize}
	  \item ... nos dispositivos m�veis;
	\end{itemize}
      \item Catalogar ferramentas mais utilizadas para testes automatizados para web:
	\begin{itemize}
	  \item ... com foco nos navegadores dos dispositivos m�veis;
	\end{itemize}
      \item Avaliar o uso das ferramentas catalogadas para testes de acessibilidade automatizados:
	\begin{itemize}
	  \item ... nos navegadores dos dispositivos m�veis;
	\end{itemize}
      \item Analisar os testes de acessibilidade automatizados nos navegadores dos dispositivos m�veis.
    \end{itemize}
  \end{exampleblock}

\end{frame}

%===============================================================================
% Cronograma
%===============================================================================
\section{Cronograma}

\begin{frame}

\frametitle{Cronograma}

\begin{scriptsize}
\begin{enumerate}
 \item Definir escopo e Cursar as disciplinas pendentes;
 \item Pesquisa bibliogr�fica;
 \item Definir e implementar testes;
 \item Escrever monografia;
 \item Revisar monografia;
 \item Concluir monografia.
\end{enumerate} 
\end{scriptsize}

% tabela do cronograma
\begin{scriptsize}
\begin{center}
 \begin{tabular}{ | c | c | c | c | c | c | c | c | c |}
   \hline
& Janeiro & Fevereiro & Mar�o & Abril & Maio & Junho & Julho & Agosto \\ 
\hline
1 & \cellcolor{gray} & \cellcolor{gray} &  &  &  &  &  &  \\
\hline
2 & \cellcolor{gray} & \cellcolor{gray} & \cellcolor{gray} & \cellcolor{gray} & \cellcolor{gray} & \cellcolor{gray} &  &  \\
\hline
3 &  &  & \cellcolor{gray} & \cellcolor{gray} & \cellcolor{gray} &  &  &  \\
\hline
4 &  & \cellcolor{gray} & \cellcolor{gray} & \cellcolor{gray} & \cellcolor{gray} & \cellcolor{gray} &  &  \\
\hline
5 &  &  &  &  &  &  & \cellcolor{gray} &  \\
\hline
6 &  &  &  &  &  &  &  & \cellcolor{gray} \\
\hline
 \end{tabular}
\end{center}
\end{scriptsize}

\end{frame}

%===============================================================================
\section{Refer�ncias Bibliogr�ficas}

  \nocite{*}

%estilos: abbrv, acm, alpha, apalike, ieeetr, plain, siam e unsrt
\bibliographystyle{abnt-alf} %abnt-alf ordenado por [nome_autor] ao inves de [numero]
\bibliography{projetoMonografia}


\end{document}